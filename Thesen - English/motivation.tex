\section{Motivation}

The content of this thesis is the development of a video game, that simulates the game table football. \textit{Table football, also known as table soccer, and known as foosball in North America, is a table-top game that is loosely based on football.[1] The aim of the game is to move the ball into the opponent's goal by manipulating rods which have figures attached}\citep{table_football} The usual table contains 4 rods on both sides, that has a total of 11 figures attached for each player.



Der Inhalt dieser Arbeit ist das Entwickeln eines Videospiel, der das Spiel Tischfußball simuliert. Tischfußball (oder Tischkicker) ist \textit{eine mit einem Spielgerät ausgeübte Sportart bzw. ein dem Fußball nachempfundenes Spiel. Ziel ist es, mit an drehbaren Griffstangen über einer rechteckigen Spielfläche angebrachten (Fußball)spielerfiguren einen Ball ins gegnerische Tor zu schießen. Ein Kickertisch ist üblicherweise mit je vier Griffstangen an den beiden Längsseiten der Spielfläche ausgestattet, an denen pro Spielpartei zusammen elf Spielfiguren angebracht sind} \citep{Tischkicker}.\\

Für die Entwicklung des Videospiels muss zuerst die Spieloberfläche modelliert werden, dazu gehört mindestens ein Kickertisch, mit 8 Stangen, an den insgesamt 22 Spielfiguren angebracht sind, zwei Tore und ein Ball. Für das Modellieren der Spieloberfläche wird für dieser Arbeit der 3D-Modellierungsprogramm Blender\footnote{\url{https://www.blender.org/}} 2.8 eingesetzt.\\

Anschließend benötigen wir eine Physik-Engine, um den Aufprall zwischen Spielfiguren und Ball zu simulieren. Für die Engine werden zwar Ziele gesetzt: Realismus und Spielbarkeit. Die Interaktionen zwischen Ball und Spielfiguren sollen möglichst realistisch sein, dennoch muss es dem Spieler möglich sein, den Ball zu kontrollieren, auch wenn letzteres etwas Training bedürft. Bei der Programmierung der Engine soll ein Kompromiss zwischen Realismus und Spielbarkeit gefunden werden, so dass ein realistisches und nutzerfreundliches Spiel entsteht.\\

Zuletzt sollen Steuerung, Spielmenü, Punktanzeige, visuelle Effekte (z.B. beim Schießen eines Tors) und Soundeffekte programmiert werden.\\

Dieses Programm soll als Basis für eine weitere Arbeit dienen, in der eine K.I. entwickelt werden sollte, die durch das Spielen gegen echten Gegnern lernt, sich durch Maschinenlernen verbessert, und dessen selbstständige Entwicklung erforscht werden kann. Außerdem kann das Spiel zur Unterhaltung dienen und sollte es kommerzialisiert werden, kann das Spiel auch eine Quelle sein, um Geld zu verdienen.