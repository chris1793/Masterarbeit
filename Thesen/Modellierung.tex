\section{Modellierung}

Für die Modellierung der Simulation würde die Software Blender 2.8 verwendet. \textit{Blender ist ein kostenloses Open Source Programm zur, unter anderen, 3D Modellierung, Animation, Simulation, Rendern und Videobearbeitung}. \citep{Blender}.\\

\todo{Richtig formulieren, Link in der Bibliographie hinzufügen}

Für die Modellierung ist es wichtig, dass die Maßen der Komponenten möglichst dem echten Spiel entsprechen, damit sich die Simulation realistisch auf dem Spieler wirkt. Allerdings gibt es für Tischkicker keine standardisierten Maßen. Es gibt sehr viele verschiedene Herstellern und die Maßen variieren dementsprechend. Für die Modellierung des Kickertisch wurden die Messungen der folgenden Webseite benutzt http://tischkicker-kaufen.org/tischkicker-masse/. Für die Stangen würde folgende Graphik benutzt: https://www.tischkicker.de/media/catalog/product/cache/1/image/800x800/9df78eab33525d08d6e5fb8d27136e95/m/a/massive-kickerstange-abmessungen.png\\

\todo{Richtig formulieren}

Das Verhältnis zwischen den Maßen der Stangen und des Tisches müsste also per Hand und nach Gefühl bestimmt werden. Diese können sich bei der Implementierung ändern, sollten dadurch Problemen auftreten.