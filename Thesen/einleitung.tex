\section{Einleitung}

%Bedeutung von Machine Learning
%Gebiet, der zurzeit sehr geforscht wird
%Automatisierung von Arbeit
%Ausprobieren und Bewertung von verschiedenen Lernverfahren.

\textit{Das Gebiet Machine Learning interessiert sich dafür, wie ein Computer bestimmte Aufgaben erlernen kann, wie zum Beispiel das erkennen von Zeichen, das Unterstützen von Diagnosen von Patienten unter schweren Erkrankungen, [...] oder das Trennen von Material entsprechend ihrer Qualität}\citep{MLDefinition}. Computer lernen auf einer ähnlichen Weise, wie wir Menschen: Als Kinder zeigen uns unsere Eltern Objekte und nennen ihre Namen. Wir als Kinder suchen zuerst nach geeignenten Merkmale, mit welchen wir ein Objekt klassifizieren können. Beispielsweise lernen wir was ein Stuhl ist und suchen nach einem Merkmal, das für die Klassifizierung ausschlaggebend ist, Die Farbe ist in diesen Fall keine gutes Merkmal, da Stühle viele verschiedene Farben haben können, dagegen ist die Form ein geeignetes Merkmal. {MLDefinition}. Je mehr Stühle wir sehen desto genauer wird unsere Einschätzung und irgendwann können wir sogar Objekte korrekterweise als Stühle erkennen, obwohl wir sie in dieser Form noch nie gesehen haben. Bei Computern sind diese Merkmale in Form von Variablen gegeben und die Information. Um einen Computer Klassifizierung beizubringen müssen wir ihnen eine Menge von Objekte der selben Klasse zum lernen geben, sogenannte \textbf{Trainingsdaten}. Computer können nicht nur Objekte durch Machine Learning klassifizieren, sondern auch Zustände und anhand dieser kann die Maschine automatisch reagieren. Ein Beispiel dafür ist das autonome Fahren. So wie wir eine Situation durch unsere Sinne erkennen und dementsprechend reagieren können, könnte eine ausgereifte künstliche Intelligenz(KI) das ebenfalls mit Hilfe von Sensoren.\\

In der Theorie könnte künstliche Intelligenz eine sehr große Menge an Arbeit ausführen, die bis jetzt nur den Mensch aufgetragen wird. Dennoch unterstützt sie uns nur bei einem sehr kleinen Anteil unserer Aufgaben. Der Grund dafür ist, dass die Technologie nicht ausgereift genug ist. Das Erschaffen einer KI ist oft zu aufwändig und das Ausführen bestimmter Arbeit kann zu Sicherheitsrisiken führen, sollte sie nicht wie gewollt funktionieren. Dennoch hat Machine Learning ein großes Potential und deshalb ist es wichtig daran intensiv zu forschen. 

Der Inhalt dieser Arbeit ist das Entwickeln einer virtuellen Simulation des Spiels Tischfußball. Tischfußball (oder auch \textit{Kicker} genannt) ist \textit{eine mit einem Spielgerät ausgeübte Sportart bzw. ein dem Fußball nachempfundenes Spiel. Ziel ist es, mit an drehbaren Griffstangen über einer rechteckigen Spielfläche angebrachten (Fußball)spielfiguren einen Ball ins gegnerische Tor zu schießen. Ein Kickertisch ist üblicherweise mit je vier Griffstangen an den beiden Längsseiten der Spielfläche ausgestattet, an denen pro Spielpartei zusammen elf Spielfiguren angebracht sind} \citep{Tischkicker}. Diese soll als eine Lernumgebung für eine künstliche Intelligenz dienen, die selbstständig spielen soll. Bei jedem Spiel gegen einen menschlichen Gegner sollen Trainingsdaten gesammelt werden, aus welchen die KI lernen und sich verbessern kann. Diese Arbeit beschäftigt sich mit dem Modellieren und Programmieren der Testumgebung. Das Entwickeln und Trainieren der KI soll in einem zukünftigen Projekt behandelt werden.\\

Zur Umgebung gehören mindestens ein Kickertisch, mit 8 Stangen an denen insgesamt 22 Spielfiguren angebracht sind, zwei Tore und ein Ball. Die Spieler für beide Teams sind auf folgende Weise aufgestellt: 1 Torhüter, 2 Verteidiger, 5 Mittelfeldspieler und 3 Stürmer. Die Spielfiguren der Simulation sollen sich durch menschliche Interaktion bewegen können, damit die KI von menschlichen Spielern lernen kann und die Bewegungen des Balls sollen durch eine Physik-Engine simuliert werden. Für das Modellieren der Spieloberfläche wird in dieser Arbeit der 3D-Modellierungsprogramm Blender\footnote{\url{https://www.blender.org/}} 2.8 verwendet.\\ 

\todo{Immer noch die richtige Blender-Version?}

Eine Physik-Engine soll anhand der physikalischen Gesetzte das Verhalten des Balls beim Interagieren mit den anderen Spielkomponenten simulieren. Realismus spielt offensichtlich eine Wichtige Rolle. Das Verhalten des Balls soll im Auge des Nutzers natürlich erscheinen, so hat es dieser leichter, die Flugbahn des Balls instinktiv zu berechnen und das Spiel zu meistern. Unter Umstände müsste allerdings ein Kompromiss zwischen Realismus und Spielbarkeit gefunden werden, denn obwohl der eigentliche Tischfußball mit dem echten Gesetzen der Physik spielbar ist, kann es beim Benutzen eines Controllers oder Maus zum Bedienen der Spielfiguren Schwierigkeiten geben den Ball zu kontrollieren. Computermäuse oder ein Controller sind ungenauer als die Stangen beim richtigen Spiel und durch diese Geräte kann der Spieler keinen Widerstand beim Aufprall mit dem Ball fühlen und dies kann sich auch auf die Spielbarkeit einwirken. Es soll mit der Engine experimentiert werden, um dem Spieler eine gewisse Kontrolle über dem Spiel zu ermöglichen, ohne es unrealistisch wirken zu lassen.\\

Als Eingabegeräte kommen Maus, Tastatur, oder Controller (wie bei den Konsolen von Playstation oder Xbox). Es sollen als Teil dieser Arbeit verschiedene Eingabemethoden ausprobiert werden und anhand von Nutzerangaben, die für ihnen beste Eingabemethode gefunden werden.\\

Zuletzt sollen Steuerung, Spielmenü, Punktanzeige, visuelle Effekte (z.B. beim Schießen eines Tores) und Soundeffekte programmiert werden. Für die Programmierung des Spiels, wird Unreal Engine 4\footnote{\url{https://www.unrealengine.com/en-US/}} verwendet.\\