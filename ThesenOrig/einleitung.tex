\section{Schriftliche Ausführung von Abschlussarbeiten}

\subsection{Allgemeine Richtlinien}

\begin{itemize}
\item Anzahl einzureichender Exemplare:
\begin{itemize}
\item Bachelor- oder Masterthesis: 3 gebundene Exemplare und in digitaler Form, 
\item Projektarbeiten: je nach individueller Absprachen.
\end{itemize}

\item Bachelor- oder Masterthesis werden beim Studierendenservice eingereicht; Projektarbeiten beim zuständigen Hochschullehrer.
\item Achten Sie auf eine termingerechte Abgabe bzw. stellen Sie rechtzeitig einen Verlängerungsantrag beim Studierendenservice.
\item Der Umfang einer Bachelorthesis sollte 40 bis 50 Seiten, der einer Masterthesis 60 bis 80 Seiten betragen und kann durch einen Anhang erweitert werden.
\item Bei Projektarbeiten sind die Absprachen mit dem Betreuer verbindlich.

\end{itemize}


\subsection{Äußere Gestaltung }

\begin{itemize}
\item Die Arbeiten sind im A4-Seitenformat, Zeilenabstand 1 1/2-zeilig mit einseitigem Ausdruck anzufertigen. Der Rand zur Heftung soll 2,5 bis 3 cm betragen, der Korrekturrand 2,5 cm.
\item Bachelor- oder Masterthesis werden mit festem A4-Einband und festem Rücken gebunden.
\item Für Projektarbeiten sind die Absprachen mit dem Betreuer bzw. der Betreuerin verbindlich.
\item Für das Lesen längerer Fließtexte eignen sich Serifenschriften wie Times New Roman oder Garamond besser als serifenlose Schriften. Wechseln Sie innerhalb Ihrer Arbeit nicht die Schriftart (auch nicht für Kapitelüberschriften und Bildunterschriften).
\end{itemize}

\subsection{Inhaltliche Gestaltung }

\textbf{Aufbau der Abschlussarbeit}
\begin{itemize}
\item Vorderer Einband mit Aufdruck („Bachelorthesis“ bzw. „Masterthesis“ sowie Vor- und Zuname)
\item Titelseite laut Muster (siehe Anhang)
\item Abstract (kurze Zusammenfassung der Aufgabenstellung und der Ergebnisse, ca. eine halbe Seite)
\item ggf. Sperrvermerk
\item Inhaltsverzeichnis mit Seitenangaben 
\item Verzeichnis der verwendeten Symbole und Abkürzungen, alphabetisch sortiert
\item Textteil 
\item Anhang
\item Quellenverzeichnis
\item Glossar, Index (optional)
\item Selbstständigkeitserklärung laut Muster
\end{itemize}

\textbf{Bilder, Grafiken und Tabellen}

\begin{itemize}
\item Abbildungen müssen mit einer Nummer und einer Bildunterschrift versehen werden. Beziehen Sie sich im Text auf die Abbildung mit Hilfe der Nummer, um die inhaltliche Zuordnung zu erleichtern; dasselbe gilt für Tabellen.
\item Grafiken müssen mit einer Legende versehen werden. Vergessen Sie nicht die Achsenbeschriftung und die Einheiten.
\end{itemize}

\textbf{Inhalt des Textteils der Arbeit}

\begin{itemize}
\item Einleitung (Umfang max. 5 Seiten) ...
\begin{itemize}
\item liefert eine Einordnung der Arbeit in das Umfeld (Wozu ist die Arbeit notwendig?),
\item erläutert die Randbedingungen der Arbeit (Wo wurde die Arbeit durchgeführt?),
\item stellt die Aufgabenstellung vor (Was soll gemacht werden?),
\item gibt einen Überblick über den Stand der Technik,
\item formuliert die Ziele der Arbeit in strukturierter Form (diese Ziele sollen in der Zusammenfassung wieder aufgegriffen werden),
\item erläutert die Abgrenzung der Arbeit, d. h. es wird kurz erwähnt, was nicht Gegenstand der Arbeit in Bezug auf angrenzende Probleme ist.
\end{itemize}

\item Hauptteil ...  
\begin{itemize}
\item soll orientiert sein an fachkundigen Lesern,
\item muss ein theoretischen Teil enthalten, der den theoretischen Hintergrund der Arbeit erläutert,
\item muss die verwendeten Methoden und Tools erläutern,
\item soll Programmablauf durch geeignete Diagramme erläutern,
\item Drucken Sie keinen kompletten Quelltext ab! Quellkodeausschnitte und Pseudocode sind besser geeignet, um die wesentlichen Punkte darzustellen. Verweisen Sie gegebenenfalls auf die beigelegte elektronische Form des Quelltextes.
\item stellt die Ergebnisse Ihrer Arbeit ggf. unter Verwendung hilfreicher Grafiken und 
Tabellen geeignet dar. 
\end{itemize}

\item Zusammenfassung (Umfang max. 4 Seiten) ...
\begin{itemize}
\item erläutert, ob Sie die Ziele, die zu Beginn formuliert wurden, erreicht haben und in welchem Umfang,
\item erläutert ggf., wie aufgetretene Probleme angegangen werden könnten,
\item kann Vorschläge zur Weiterführung liefern, 
\item enthält keine Ergebnisse, die nicht bereits im Hauptteil erwähnt wurden.
\end{itemize}

\end{itemize}

\textit{Bitte bedenken Sie bei der Abfassung Ihrer Arbeit, dass häufig zunächst oder ausschließlich Abstract, Einleitung und Zusammenfassung gelesen werden. Diese Teile sind also die Aushängeschilder Ihrer Arbeit.}

\textbf{Anhang}
\begin{itemize}
\item Es sollte nur solches Material in den Anhang aufgenommen werden, das für die Arbeit wichtig ist und worauf im Textteil verwiesen wird.
\item Im Anhang wird Material zusammengefasst, das im fortlaufenden Text nicht anzuordnen ist, da es dort den Fluss der Darstellung stören würde (umfangreiche Bilder, umfangreiche Diagramme, umfangreiche Tabellen oder Tabellen von Werten, die im Textteil grafisch dargestellt werden, Dokumentationen, usw.).
\item Sehr umfangreiche Materialien, die wesentlicher Teil der Arbeit sind, sollten elektronisch der Arbeit beigelegt werden.

\end{itemize}


\textbf{Quellenverzeichnis}

Die einzelnen Literaturangaben sind so abzusetzen, dass sie problemlos als eigenständige Angaben erkannt werden können, z.B. durch Leerzeilen oder durch Einrücken. Es empfiehlt sich, Literaturangaben durch Bezeichnungen in eckigen Klammern, z.B. [2] oder [Eul08] zu ergänzen, auf die Sie sich im Text beziehen.\\ \\

Aufbau der Literaturangaben:
\begin{itemize}
\item bei Artikeln: Autor, weitere Namen, "Titel der Literaturstelle", Zeitschrift, Jahrgang, Jahr, laufende Nummer, Seitenzahlen
\item bei Büchern: Autor, ggf. Herausgeber, "Buchtitel", Auflage, Verlag, Erscheinungsort, Erscheinungsjahr, ggf. Seite
\item bei Gesprächsnotizen: Firma, Ort, Datum, Gesprächsteilnehmer
\item bei unveröffentlichten Quellen: Autor, "unveröffentlichter Bericht" oder ähnliches, Firma, Ort, Jahr
\item Anonyme Artikel, Firmenschriften, Prospekte ohne Angabe eines Verfassers erhalten im Verzeichnis anstelle des Verfassernamens: o. V. 
\item Bei Internetquellen: URL mit Datum, Kopie auf Datenträger 
\end{itemize}

\textbf{Weitere Hinweise}

\begin{itemize}
\item Achten Sie auf einen "roten Faden" in Ihrer wissenschaftlichen Arbeit. Formulieren Sie die durchdachten Inhalte in einer unpersönlichen Form, nutzen Sie einen strukturierten Textfluss.
\item Setzen Sie auflockernde und bereichernde Gestaltungsmittel ein (Bilder, Tabellen, Diagramme, Fotos, usw.).
\item Quellen müssen  eindeutig kenntlich gemacht werden; dies gilt auch für fremden Programmcode. Beachten Sie, dass ein Plagiatsvorwurf zur Aberkennung Ihres akademischen Grades führen kann.
\end{itemize}

\section{Elektronische Ausführung der Arbeit}

Die gedruckte Form der Arbeit muss um einen elektronischen Datenträger ergänzt werden. Die Datenträger sind eindeutig zu beschriften und mit Einstecktaschen an der Arbeit zu befestigen. \\  \\
Der elektronische Datenträger soll enthalten:

\begin{itemize}
\item ein Inhaltsverzeichnis des Datenträgers,
\item das entwickelte Programm mit ReadMe-Datei zur Installation und notwendigen Hilfsprogrammen,
\item ein Benutzerhandbuch zum Programm,
\item alle gut kommentierten Quelltexte des Programms,
\item alle zitierten Web-Seiten.
\end{itemize}

Verwenden Sie möglichst system- und umgebungsunabhängige Dateiformate. \\ \\
In der begleitenden elektronischen Form Ihrer Arbeit, sind Textmarken für eine Navigation wünschenswert, eine direkte HTML-Linkverbindung im Anhang ist hilfreich.
\\ \\
\textbf{Bewertung der Abschlussarbeit}
Folgende Aspekte spielen bei der Bewertung von Abschlussarbeiten eine Rolle:

\begin{itemize}
\item Die Arbeit lässt spezielle Kenntnisse im Studienfach erkennen.
\item Die Arbeit entspricht den Anforderungen der Ausschreibung.
\item Die Arbeit behandelt das theoretische Umfeld der Aufgabenstellung und gibt einen Überblick über den Stand der Technik.
\item Eine wissenschaftliche Arbeitsweise ist erkennbar. Hierzu gehören Literatur-Recherche, Erkennen von Problemen und ihrer Struktur, Untersuchung verschiedener Lösungsansätze. 
\item Die Ergebnisse sind verständlich dargestellt.
\item Der Kandidat bzw. die Kandidatin hat bei der Anfertigung der Arbeit zielstrebiges, selbstständiges Arbeiten gezeigt.
\item Die Form der Arbeit ist angemessen: gute Gliederung, Orthographie und Grammatik korrekt, angemessene Verwendung von Fachsprache, hilfreiche Abbildungen mit Bildunterschriften und Erläuterungen, korrekte Literaturangaben.
\end{itemize}


\section{Kolloquium}
Das Kolloquium stellt die Verteidigung Ihrer Abschlussarbeit vor einer Prüfungskommission dar.

\begin{itemize}
\item Kolloquien sind in der Regel am Fachbereich öffentliche Veranstaltungen und dauern in der Regel insgesamt 60 Minuten.
\item In Ihrem Kolloquium stellen Sie in einem 20-minütigen Vortrag Ihre Ergebnisse der Abschlussarbeit vor. Halten Sie diese Zeitvorgabe unbedingt ein.
\item Für den Vortrag sind elektronische Präsentationen (z.B. PowerPoint) empfohlen.
Empfohlene Gliederung (die Gliederung soll präsentiert werden)
\begin{itemize}
\item Einleitung: Thematik, Stand der Technik, Aufgabenstellung
\item Methoden
\item Ergebnisse
\item Zusammenfassung
\end{itemize}

\item Sprechen Sie möglichst frei sowie laut und deutlich. Verwenden Sie möglichst wenig Abkürzungen. Erläutern Sie speziellere Fachbegriffe und reduzieren Sie das Spezialvokabular auf das unbedingt Notwendige.

\end{itemize}

\section{Poster}
Es ist am Fachbereich Informatik eine gute Tradition geworden, dass der Absolvent oder die Absolventin über seine bzw. ihre Arbeit ein A0-Poster anfertigt. Dieses Poster kann im Kolloquium hilfreich sein und wird bei guter fachlicher Qualität im Fachbereich ausgehängt. \\ \\
Inhalt des Posters (übersichtlich anordnen)

\begin{itemize}
\item Thema 
\item Verfasser(in), Betreuer(innen) 
\item Auszüge aus Einleitung und Zusammenfassung der Arbeit 
\item repräsentative Ergebnisse der Arbeit 
\item interessante Bilder mit erläuterndem Text bei ansprechender grafischer Gestaltung 

\end{itemize}
